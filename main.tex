\documentclass[12pt]{article}
\usepackage[utf8]{inputenc}
\usepackage[brazil]{babel}
\usepackage{graphicx}
\usepackage{geometry}
\usepackage{setspace}
\usepackage{titlesec}
\usepackage{fancyhdr}

% Configuração da página
\geometry{a4paper, margin=2.5cm}
\setstretch{1.5}
\titleformat{\section}{\normalfont\Large\bfseries}{\thesection}{1em}{}

% Variáveis personalizáveis
\newcommand{\nomeAluno}{NOME DO ALUNO}
\newcommand{\matricula}{MATRÍCULA}
\newcommand{\nomeProfessor}{NOME DO PROFESSOR}
\newcommand{\disciplina}{NOME DA DISCIPLINA}
\newcommand{\tituloTrabalho}{TÍTULO DO TRABALHO}
\newcommand{\dataEntrega}{DATA DE ENTREGA}
\newcommand{\logoFaculdade}{logo_ufsc.png} % Substitua pelo caminho do seu logo

% Início do documento
\begin{document}

% Capa
\begin{titlepage}
    \begin{center}
        \includegraphics[width=0.3\textwidth]{\logoFaculdade} \\[1cm]
        {\large UNIVERSIDADE FEDERAL DE SANTA CATARINA}\\[0.5cm]
        {\large \disciplina}\\[3cm]
        {\LARGE \textbf{\tituloTrabalho}}\\[3cm]
        \begin{flushright}
            \textbf{Aluno(a):} \nomeAluno\\
            \textbf{Matrícula:} \matricula\\
            \textbf{Professor(a):} \nomeProfessor\\
            \textbf{Data:} \dataEntrega\\
        \end{flushright}
        \vfill
        {\large Local – Ano}
    \end{center}
\end{titlepage}

% Sumário
\tableofcontents
\newpage

% Introdução
\section{Introdução}
\subsection{Contextualização}
Texto de exemplo para contextualização.

\subsection{Objetivos}
Texto de exemplo para os objetivos.

% Desenvolvimento
\section{Desenvolvimento}
\subsection{Revisão de Literatura}
Texto de exemplo com referência~\cite{livro_exemplo}.

\subsection{Metodologia}
Texto de exemplo para metodologia.

\subsection{Análise dos Resultados}
Texto de exemplo para análise.

% Conclusão
\section{Conclusão}
\subsection{Resumo dos Resultados}
Texto de exemplo para resumir os achados.

\subsection{Trabalhos Futuros}
Texto de exemplo para sugestões futuras.

% Referências
\addcontentsline{toc}{section}{Referências}
\begin{thebibliography}{9}

\bibitem{livro_exemplo}
Autor Exemplo.
\textit{Título do Livro}.
Editora Exemplo, 2020.

\bibitem{artigo_exemplo}
Autor do Artigo.
\textit{Título do Artigo}.
Revista Acadêmica, v.10, n.2, p.100-110, 2022.

\end{thebibliography}

\end{document}